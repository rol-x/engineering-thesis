\chapter{Specification}
\label{ch:specification}
How will the topic of this thesis be implemented? How do I combine the aforementioned technologies? What is the actual example that I'm using? Describe the specification from the outside (helicopter view).


\section{Functional requirements}
Wymagania funkcjonalne.


\section{Non-functional requirements}
Wymagania niefunkcjonalne.


\section{Problem domain}

\subsection{Trading card games}
Trading card games (\textit{TCG}), also known as collectible card games (\textit{CCG}), are types of card games combining the elements of strategic gameplay and features of trading cards.
The first TCG was released in 1993 under the name Magic: The Gathering (\textit{MTG}). The game, released by an eight-person company, was an overnight success, with over 10 million cards sold in just 6 weeks.
Two years later, this basement business became a gaming corporation. Today, \textit{MTG} is among the most popular TCGs with roughly 35 million players as of December 2018 \cite{magicTheGathering}. \par
During the game, the goal of each player is to reduce the opponent's life points by strategically playing cards from the hand, before the other one succeeds. However, each player can compile their own deck of 60\footnote{Some variations require a deck of 40 cards, which are selected from a random pool of cards} cards out of the thousands available. This makes every gameplay unique on a level which is fundamentally different from the classic cards. One doens't have to be an expert in the field to understand that the more cards one has, especially good cards, the higher the chances of winning. Thus, trading the cards becomes an aspect as crucial as the strategy itself. \par
One of the places where \textit{MTG} cards can be bought is \url{www.cardsmarket.com/} --- an online market with a myriad of cards from the game listed for sale, by users from around the world, majority of which lives in Europe. Users are divided into three categories: \textit{Amateur}, \textit{Professional} and \textit{Powerseller}, depending on their setup and \textit{modus operandi} (individuals, zealous hobbyists, card stores). To spend the least amount of money while collecting the most wanted cards, i.e. to optimize the shopping, one would have to analyze thousands of bits of data, from the average prices of all cards, to the velocity of sales and restocking the virtual shelves.

\subsection{Decision optimization}
Describe the cards (with their statistics --- static entity, dynamic entity). Also sellers and sale offers.
Describe what we want to do with the gathered data and to what means.


\section{Data warehouse modelling}
Describe how is the data structured, based on the entities mentioned before. Show how does the data warehouse look like including all steps, explain why is it like this. Mention DWDMS and the outcome of that project, as well as differences between it and this thesis.