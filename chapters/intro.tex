\chapter{Introduction}
\label{ch:intro}
The subject of this engineering thesis is a standalone data warehouse built using multiple Docker containers. The thesis describes how the data is gathered, processed and analyzed, and what components make up each container, as well as their mutual coordination. Additionally, practical results of the project are presented in form of answers to real-world questions, together with steps taken to achieve performance optimization and code robustness. \par
The basic assumptions of the system are the following:
\begin{enumerate}
    \item The application is portable, standalone and platform-independent.
    \item The application requires the minimal amount of setup in order to run.
    \item The code is robust in recovering from errors and able to run continuously.
    \item Proper programming practices are followed whenever it is possible.
\end{enumerate}


\section{Scope of the work}
The work done in relation to one of the containers in this project is partially based on a group project\footnote{Group members: Jakub Sieńko, Kacper Garcon and the author, collaborating during an university course \textit{Data Warehouses and Data Mining Systems}}, concerned with the acquisition of data from a web service in order to answer simple questions with SQL queries. For that project, the author of this thesis implemented the data gathering part using Python language with Selenium module. This individual work has been transfered and heavily modified in order to compose a container-based solution, with other containers written by the author from scratch for the purpose of this thesis. \par
In totality, the following parts were implemented:
\begin{itemize}
    \item data gathering container,
    \item database manager container,
    \item data mining container,
    \item logging service,
    \item flag management service,
    \item database management service,
    \item web Selenium-based service,
    \item web data transformation service,
    \item data mining service.
\end{itemize}
Additionally, the author researched...


\section{Thesis contents}
Chapter (\ref{ch:basis}) is concerned with the changes in technology leading to various options of exploiting data availability on the Internet, while also deliniating the tools which are at the core of this project. It also explains the motivation for choosing such thesis topic. \par
Chapter (\ref{ch:specification}) describes the domain of the problem, functional and non-functional requirements pertaining to it, as well as the structure of the data warehouse and the data inside it. \par
Chapter (\ref{ch:architecture}) contains technical information about the orchestration of the containers, composition of the services and an overview of the data pipeline. \par
Chapters (\ref{ch:gathering}), (\ref{ch:database}) and (\ref{ch:mining}) are centered around the three containers, taking a deep dive inside each of them and following the data on its path from the website to market reports. \par
Chapter (\ref{ch:results}) reflects on the practical results generated as a result of the data analysis, while constrasting them with initially posed questions. Additionally, the chapter describes the performance of the code under normal and abnormal conditions, and how the project changed with time. \par
Chapter (\ref{ch:summary}) summarized the effects achieved during the implementation of the system and formulated conclusions about possibilities of using containers to create and manage data warehouses.