\newpage \vspace{-5cm}
\thispagestyle{empty}

% Title page

\begin{onehalfspacing}
\begin{center}

\centering
\includegraphics[keepaspectratio, scale=0.5]{./figures/polsl_eng_notext.jpg} \\[.8cm]

{\fontsize{17.28}{16}\selectfont
\textsc{Silesian University of Technology\\[.3cm]
Faculty of Automatic Control, Electronics and Computer Science \\[.8cm]
Control, Electronic and Infromation Engineering \\[2cm]}
\textbf{Engineering thesis}\\[1.6cm]}

\LARGE{Containerization of data warehouse creation and management processes} \\[2.4cm]

\large
\begin{flushleft}
Author: Karol Latos  \\
Supervisor: dr inż. Anna Gorawska \\
\end{flushleft}

\vspace*{\fill}
Gliwice, February 2022
\end{center}
\end{onehalfspacing}

% Blank page

\newpage
\thispagestyle{empty}
\mbox{}
\clearpage

\thispagestyle{plain}
\vspace*{4.8cm}
\begin{center}
    \textbf{Abstract}
\end{center}
Data warehouses are powerful systems responsible for gathering, storing, managing and transforming the data related to a single problem domain. As the data is processed through the pipeline to generate the wanted results, many different parts of the system are involved. To achieve cross-platform portability, separate independent stages and make the application standalone, containerization of the consecutive steps is introduced. This thesis describes a Docker-based data warehouse management system, related to an online card market data, aiming at providing the user with useful knowledge to improve their buying decisions. \par \noindent
\textbf{Keywords: }\;data warehouse,\;containerization,\;data cleaning,\;data mining

% Table of contents
\pagenumbering{Roman}
\setcounter{page}{0}
\tableofcontents

\pagestyle{fancy}
